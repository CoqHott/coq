\achapter{\protect{Miscellaneous extensions}}
%HEVEA\cutname{miscellaneous.html}

\asection{Program derivation}

Coq comes with an extension called {\tt Derive}, which supports
program derivation. Typically in the style of Bird and Meertens or
derivations of program refinements. To use the {\tt Derive} extension
it must first be required with {\tt Require Coq.Derive.Derive}. When
the extension is loaded, it provides the following command.

\subsection[\tt Derive \ident$_1$ SuchThat \term{} As \ident$_2$]
       {\tt Derive \ident$_1$ SuchThat \term{} As \ident$_2$\comindex{Derive}}

The name $\ident_1$ can appear in \term. This command opens a new
proof presenting the user with a goal for \term{} in which the name
$\ident_1$ is bound to a existential variables {\tt ?x} (formally,
there are other goals standing for the existential variables but they
are shelved, as described in Section~\ref{shelve}).

When the proof ends two constants are defined:
\begin{itemize}
\item The first one is name $\ident_1$ and is defined as the proof of
  the shelved goal (which is also the value of {\tt ?x}). It is
always transparent.
\item The second one is name $\ident_2$. It has type {\tt \term}, and
  its body is the proof of the initially visible goal. It is opaque if
  the proof ends with {\tt Qed}, and transparent if the proof ends
  with {\tt Defined}.
\end{itemize}

\Example
\begin{coq_example*}
Require Coq.derive.Derive.
Require Import Coq.Numbers.Natural.Peano.NPeano.

Section P.

Variables (n m k:nat).

\end{coq_example*}
\begin{coq_example}
Derive p SuchThat ((k*n)+(k*m) = p) As h.
Proof.
rewrite <- Nat.mul_add_distr_l.
subst p.
reflexivity.
\end{coq_example}
\begin{coq_example*}
Qed.

End P.

\end{coq_example*}
\begin{coq_example}
Print p.
Check h.
\end{coq_example}

Any property can be used as \term, not only an equation. In
particular, it could be an order relation specifying some form of
program refinement or a non-executable property from which deriving a
program is convenient.
